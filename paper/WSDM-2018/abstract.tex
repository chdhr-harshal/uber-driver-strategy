%!TEX root = main.tex

\begin{abstract}
On-demand ridesharing platforms like Uber and Lyft play an important role in urban transportation, 
  enabling car owners to become drivers for hire with minimal overhead.
One natural optimization problem in this domain has focused on optimal route planning of a deployed
  fleet of vehicles, with objective functions geared largely towards improving the overall service 
  to passengers or minimizing total resource utilization (fuel consumption, miles driven, etc.).
But much less emphasis has been placed on optimization for individual drivers.
While some individuals drive opportunistically either as their schedule allows or on a fixed 
  schedule, strategic behavior regarding when and where to drive can substantially increase
  driver income. In this paper, we formalize the problem 
  of devising a driver strategy so that he maximizes his expected earnings.
We devise a series of dynamic-programming algorithms that show how to solve this
problem under different set of actions available to the drivers.
We also develop a framework for analyzing the sensitivity of our results to the
input data. 
Our experiments use a newly collected dataset, which combines the NYC taxi rides
with data we collected using the Uber API. The latter has
information about profits that would be made by Uber drivers for the same ride at the same 
time and day.
In an extensive experimental evaluation, we demonstrate
that our methods provide useful insights regarding passenger and driver behaviors.
\end{abstract}
