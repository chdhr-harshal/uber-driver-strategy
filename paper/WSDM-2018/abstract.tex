%!TEX root = main.tex

\begin{abstract}
On-demand ridesharing platforms like Uber and Lyft are helping reshape urban 
transportation, 
by enabling car owners to become drivers for hire with minimal overhead.
%One natural research direction in this domain has focused on optimization of a fleet of
%  vehicles, with objective functions in route planning geared towards 
%  minimizing total resource utilization 
%%(fuel consumption, miles driven, etc.) 
%  or improving overall service to passengers. 
Although there are a lot of studies that view ridesharing platform holistically as
supply-demand economies, little
emphasis has been placed on optimization for the individual, self-interested 
  drivers that currently comprise these fleets.
While some individuals drive opportunistically either as their schedule allows or on a fixed 
  schedule, we show that strategic behavior regarding when and where to drive can substantially 
  increase driver income. 
In this paper, we formalize the problem of devising a driver strategy to maximize expected 
  earnings, describe a series of dynamic programming algorithms to solve these problems
  under different sets of modeled actions available to the drivers, and exemplify
  the models and methods on a large scale simulation of driving for Uber in NYC.
To conduct our experiments, we use a newly-collected dataset that combines the NYC taxi 
  rides dataset along with Uber API data, to build time-varying traffic and payout matrices
  for a representative six-month time period in greater NYC.
From this input, we can reason probabilistically about prospective itineraries and payoffs.
Moreover, the framework enables us to rigorously reason about
  and analyze the sensitivity of our results to perturbations in the input data. 
Among our main findings is that repositioning throughout the day is key to maximizing
  driver 
  earnings, whereas `chasing surge' is typically misguided and sometimes a costly move.
\end{abstract}
