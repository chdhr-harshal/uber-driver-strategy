%!TEX root = main.tex

\section{Conclusions}
\label{sec:conclusions}

The challenge of how to maximize one's individual earnings as a driver for a 
ridesharing platform like Uber or Lyft is a pressing question that millions of micro-entrepreneurs 
across the world now face.  Anecdotally, many drivers spend a great deal of time 
strategizing about where and when to drive.  However, to date, drivers are either
self-taught and use heuristics of their own devising, or learn from one another.
Indeed, rumors suggest that some drivers collude in attempts to spike surge prices.
In this work, we confirm the power of strategic driving behavior, by simulating 
key points in the strategy space optimized against realistic data-driven projections
of ridership in the NYC area.  Our first key takeaway is that a naive driver,
armed with no data, and driving a 9-5 random walk schedule, is leaving 
roughly a 50\% hourly pay raise on the table by not driving more strategically 
in terms of time and location.  In contrast, a data-savvy driver armed with
good historical data can build a forecast and optimal contingency plans for 
an upcoming week of driving with a couple hours of computational overhead using 
our dynamic programming algorithms, with provable resilience to input
uncertainty.  Our experimental and simulation results yield insights into the
structure of highly optimized schedules, including relatively frequent relocation,
working at specific peak periods, and opportunistically taking advantage of 
surges when the time is ripe.  

An obvious limitation of our work is that it is tailored to the setting when the 
methods are employed by self-interested individuals.  If and when a 
significant percentage of the labor supply employs sophisticated optimization methods
for driving, one would need to consider different strategies that lead either to mixed 
equilibria, or achieve other global objective functions, as opposed to the simple 
greedy approach we invoke here.  Indeed, in the long run, as drivers for ridesharing 
platforms like Uber and Lyft are put out of work by fleets of autonomous vehicles, 
the formulation and solution of new sets of optimization problems along those lines
are likely to become relevant as well.



