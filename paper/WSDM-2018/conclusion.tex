%!TEX root = main.tex

\section{Conclusions}
\label{sec:conclusions}
In this paper, we focused on the problem of 
maximizing a driver's individual earnings on ride-hailing platforms like Uber or Lyft.
Our work confirms the power of strategic driving behavior %, by simulating 
%key points in the strategy space optimized against 
using realistic data-driven projections
of ridership in the NYC area.  Our first key takeaway is that a naive driver,
armed with no data, and driving a 9-5 random walk schedule, is leaving 
roughly a 50\% pay raise on the table by not driving more strategically. 
%in terms of time and location.  
In contrast, a data-savvy driver armed with
good historical data can build a forecast and optimal contingency driving plans 
with little computational overhead using 
our dynamic programming algorithms, with provable resilience to input
uncertainty.  Our experimental results yield insights into the
structure of highly-optimized schedules, including relatively frequent relocation,
working at specific peak periods, and taking advantage of 
surges when the time is ripe.  

An obvious limitation of our work is that it is tailored to the setting when the 
methods are employed by self-interested individuals.  If a 
significant percentage of the labor supply employs sophisticated optimization methods
for driving, one would need to consider different strategies that achieve equilibria
or other global objectives. %, as opposed to the simple 
%greedy approach we invoke here.  
Indeed, in the long run, as drivers for ride-hailing 
platforms like Uber and Lyft are put out of work by fleets of autonomous vehicles, 
the formulation and solution of new sets of optimization problems along those lines
are likely to become relevant as well.



