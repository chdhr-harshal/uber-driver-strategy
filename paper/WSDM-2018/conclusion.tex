%!TEX root = main.tex

\section{Conclusions}
\label{sec:conclusions}

In the long run, drivers for platforms like Uber and Lyft will be put 
out of work by autonomous fleets, which will perhaps necessitate the 
formulation and solution of a new set of optimization problems from those 
we consider here.  But in the meantime, the challenge of how to maximize 
one's individual earnings as a driver is a pressing question that now
millions of micro-entrepreneurs face.  Anecdotally, many drivers spend a 
great deal of time strategizing where and when to drive, albeit using 
self-taught heuristics. Rumors go so far as to suggest collusive behavior 
amongst drivers in attempts to spike surge prices.
In our work, we confirm the power of strategic behavior, by simulating 
key points in the strategy space optimized against realistic data-driven projections
of ridership in the NYC area.  Our first key takeaway is that a naive driver,
armed with no data, and driving a 9-5 random walk schedule, is leaving 
roughly a 50\% hourly pay raise on the table by not driving more strategically 
in terms of time and location.  In contrast, a data-savvy driver armed with
good historical data can build a forecast and optimal contingency plans for 
an upcoming week of driving with a couple hours of computational overhead using 
our dynamic programming algorithms, with provable resilience to input
uncertainty.  Our experimental and simulation results yield insights into what
those optimized schedules look like, including relatively frequent relocation,
working at specific peak periods, and opportunistically taking advantage of 
surges when the time is ripe.



