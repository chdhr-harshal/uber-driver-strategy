%!TEX root = main.tex

\section{Introduction}
\label{sec:introduction}

The proliferation of on-demand ridesharing platforms like Lyft and Uber
have changed the way people move within urban environments. 
In last two years alone, number of daily trips with using ridesharing platforms like 
Uber and Lyft in NYC have grown five-fold to about 350,000 trips per day. Over 65,000 drivers drive in the streets of NYC as Uber or Lyft drivers.
Explosive growth of these ridesharing platforms have created a supply-and-demand environment that offers 
plenty opportunities for academic research that focuses on pricing mechanisms
as well as algorithms for matching drivers to 
customers~\cite{banerjee2015pricing,ozkan2016dynamic}.


Although those studies view ridesharing platforms holistically, little work
has been done on optimizing strategies for drivers. 
However, the challenge of how to maximize one's individual earnings as a driver for a 
ridesharing platform like Uber or Lyft is a pressing question that millions of micro-entrepreneurs 
across the world now face.  Anecdotally, many drivers spend a great deal of time 
strategizing about where and when to drive.  However, to date, drivers are either
self-taught and use heuristics of their own devising, or learn from one another.
Indeed, rumors suggest that some drivers even collude in attempts to induce spikes in surge prices that they can then exploit.
To date there is only  some 
popular-press articles that offer(often contradictory) advice to ridesharing drivers how to maximize their earnings, mostly via chasing surge~\cite{dont,tips}. 

 In this paper, we formalize the problem of devising a driver strategy to maximize expected 
 earnings, describe a series of dynamic programming algorithms to solve this problem
 under different sets of modeled actions available to the drivers. 
 Moreover, the framework we propose enables us to rigorously reason about  and analyze the sensitivity of our results to perturbations in the input data. 
 Thus, we can justify the proposed strategies even under some uncertainty level in the
 collected data. 
  
We exemplify our  results with a large-scale simulation of driving for Uber in NYC. For this simulation we assemble a new dataset that uses both the publicly available NYC taxi rides 
dataset~\footnote{\url{http://www.nyc.gov/html/tlc/html/about/trip_record_data.shtml}} as well as calls to the Uber API. From the former we obtain information about trips that happen between different NYC zones. From the latter we obtain pricing as well as traffic-time information for those trips. This newly-collected dataset is of independent interest and can itself be used for a multitude of other studies.

Our experiments with our methods on this dataset demonstrate the following findings:
being strategic about the rides they pick, the areas they focus on picking up riders
and the times they work can increase a driver's impact more than 100\%  when compared
to a naive optimization strategy.
Moreover, we show that this pronounced difference between the earnings obtained by the 
two strategies is even true when there is a large uncertainty in the input data. This essentially means that our results are not an artifact of the dataset we collected, but rather they generalize well.
Finally, our experiments with naive surge-chasing strategies show that blindly chasing price surge not only does it not lead to significant earnings, but it can even harm the drivers, who end up greedily searching for the high-search area without strategizing appropriately.