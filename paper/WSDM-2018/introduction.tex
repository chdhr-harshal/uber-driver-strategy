%!TEX root = main.tex

\section{Introduction}
\label{sec:introduction}

The proliferation of on-demand ride-hailing platforms like Lyft and Uber
  has begun to fundamentally change the nature of urban transit. 
In the last two years alone, the number of daily trips using ride-hailing 
  platforms like Uber and Lyft in NYC has grown five-fold, 
  to about 350,000 trips per day. 
Today, over 65,000 drivers drive on the streets of NYC as Uber or Lyft drivers.
The explosive growth of these ride-hailing platforms has motivated a wide
  array of questions for academic research at the intersection of computer
  science and economics, ranging from the design of effective pricing mechanisms, 
  to equilibrium analysis, to the design of reputation management systems for 
  drivers, to algorithms for matching drivers with 
  customers, as we discuss in our related work section.
%%~\cite{banerjee2015pricing,ozkan2016dynamic}.

While these studies consider the study of ride-hailing platforms holistically, 
   little work has been done on optimizing strategies for individual drivers. 
Nevertheless, the challenge of how to maximize one's individual earnings as a driver for a 
ride-hailing platform like Uber or Lyft is a pressing question that millions of micro-entrepreneurs 
across the world now face.  Anecdotally, many drivers spend a great deal of time 
strategizing about where and when to drive.  However, drivers today are 
self-taught, using heuristics of their own devising or learning from one another, 
and employ relatively simple analytics dashboards such as SherpaShare.
Indeed, rumors suggest that some drivers even collude in attempts to induce spikes in surge prices that they can then exploit.
But in terms of concrete guidance, to date, there are only articles in the
  popular press and on blogs that offer (often contradictory) advice to ride-hailing drivers 
  how to maximize their earnings~\cite{dont,tips,sherpashareNYT}.

In this paper, we formalize the problem of devising a driver strategy to maximize expected 
 earnings and describe a series of dynamic programming algorithms to solve this problem
 under different sets of modeled actions available to the drivers. 
Our strategies take as input a detailed model of city-level data that constitutes a 
  fine-grained weekly projection of forecasted demand for rides, comprising 
  predicted spatiotemporal distributions of source-destination pairs, driver payments,
  transit times, and surge multipliers. 
The optimization framework we propose not only produces contingency plans in the form of
  highly optimized driving schedules and real-time in-course corrections to drivers, but 
  also enables us to rigorously reason about and analyze the sensitivity of our output 
  results to perturbations in the input data.  
Thus, we can justify the proposed strategies even under an uncertainty level in the
  collected data and the data model itself.
  
We then exemplify our  results with a large-scale simulation of driving for Uber in NYC. 
For this simulation, we assemble a new dataset that uses both the publicly available NYC taxi rides 
dataset~\footnote{\url{http://www.nyc.gov/html/tlc/html/about/trip_record_data.shtml}} as well as calls to the Uber API.
From the former, we obtain information about over 200,000 taxi rides that occurred between different NYC zones. 
From the latter, we obtain representative pricing and traffic-time information for those trips, were they to reoccur on Uber.
From this dataset, we construct a mathematical model to produce input to our algorithms. 
However, we view the dataset, which we plan to release, to be of independent interest that could subsequently 
be used for a multitude of other studies.

Our experiments with our methods on this dataset demonstrate the following findings.
Being strategic about the areas they focus on picking up riders and the times they work, 
drivers can significantly increase their income, sometimes by
as much as 1.5x, when compared to a naive optimization strategy.
Moreover, we show that a pronounced difference between earnings obtained 
holds even when there is large uncertainty in the input data. 
We argue that our results are therefore not purely an artifact of the NYC dataset we employ, 
  but also have high potential to generalize.
Finally, our experiments show that naively chasing surging prices does not typically lead 
  to significant earnings gains, but it can actually introduce large opportunity costs, as drivers
  waste time driving to subsiding surges. 
