%!TEX root = main.tex

\section{Related Work}
\label{sec:related_work}
\textbf{Works on general taxi cabs:}

1. \citeauthor{camerer1997labor} - seminal paper. (i) `inexperienced' cabdrivers make labor supply decisions ``one day at a time'' instead of substituting labor and leisure across multiple days. (ii) set a loose daily income target and quit working once they reach that target.

2. \citeauthor{yang1998network} - first in the series of works modeling taxi utilization and movement to find that higher utilization leads to longer waiting times for customers.

3. \citeauthor{wong2001modeling} - second paper, extends previous paper to incorporate effects of congestion and customer demand elasticity. 

4. \citeauthor{yang2002demand} - third paper, uses network model to describe demand and supply equilibrium of taxi services under fare structure and fleet size regulation in competitive / monopoly market.

5. \citeauthor{bailey1987simulation} - directed at understanding the dynamic interaction between demand, service rates and policy alternatives. Finds that customer waiting time is insensitive to changes in demand, but highly sensitive to changes in taxi fleet size.

6. \citeauthor{qin2017mining} - explore the factors affecting driver incomes with quantitative estimates using GPS traces of over 167 million trips in Shanghai.

7. \citeauthor{rong2016rich} - MDP to increase the revenue per unit time of taxi drivers. They study the relocate action from our strategy.

\textbf{Works on taxi routing:}

8. \citeauthor{maciejewski2013simulation} - optimizes taxi routing by generating demand and congested network simulation. Defines online and offline taxi dispatching strategies and evaluate them. `No-scheduling' strategy works well under low demand but deteriorates under heavy load.

9. \citeauthor{nunes2011taxi} - formulates a TSP problem to find best route for a taxi company to satisfy demand.

\textbf{Works on ride-sharing:}

10. \citeauthor{agatz2012optimization} - outline optimization challenges in developing technology to support ride-sharing services. Survey of operations research papers in the domain.

11. \citeauthor{santos2013dynamic} - Prove that problem of maximizing shared trips within a fixed time window to minimize shared expenses is NP-Hard and a propose heuristic solution.

12. \citeauthor{jung2013design} - Simulated Annealing algorithm to maximize occupancy and minimize travel times (by minimizing passenger detours) in shared-ride concept.

\textbf{Works on ride-sharing vehicle routing:}

13. \citeauthor{lin2012research} - simulated annealing algorithm to optimize routing of ride-sharing taxi to minimize operating costs while maximizing customer satisfaction.

\textbf{Strategic behavior:}

14. \citeauthor{shi2016optimization} - maximizes social welfare and optimizes allocation of taxi market resources. Also analyzes strategic behavior of passengers who may join or drop out of system based on their social welfare threshold.

\textbf{Uber related works:}

15. \citeauthor{hall2016analysis} - Drivers attracted to Uber platform due to flexibility it offers, level of compensation, earnings per hour do not vary much with number of hours worked. 

16. \citeauthor{chen2016dynamic} - Show a causal relationship that drivers on Uber respond to `surges' by driving more during high surge times, in contrast to \citeauthor{camerer1997labor} which says that drivers driver until they achieve earnings goals.

17. \citeauthor{banerjee2015pricing} - Study complex dynamic pricing strategies for ride-sharing platforms (Lyft) using a queuing-theoretic economic model. They show the dynamic pricing is robust to changes in system parameters, even if it does not achieve higher performance than static pricing. 

18. \citeauthor{ozkan2016dynamic} - Strategic matching between Uber or Lyft's supply and demand. Matching based on parameters like customer/driver arrival rates, willingness of customers to wait and time-variance can achieve better performance than naively matching passenger with closest driver.

\textbf{Media and Press articles:}

1. How Uber Uses Psychological Tricks to Push Its Drivers' Buttons\footnote{\url{https://www.nytimes.com/interactive/2017/04/02/technology/uber-drivers-psychological-tricks.html}}

2. Advice For New Uber Drivers- Don't Chase The Surge! -Rideshare guy\footnote{\url{http://maximumridesharingprofits.com/advice-new-uber-drivers-dont-chase-surge/}}

3. Chasing the Surge: 3 Tips for Maximizing Uber Earnings\footnote{\url{https://pjmedia.com/lifestyle/2016/07/19/chasing-the-surge-3-tips-for-maximizing-uber-earnings/1/}}: Some other articles suggest chasing the surge.

4. Ride-Hailing Drivers Are Slaves to the Surge\footnote{\url{https://www.nytimes.com/2017/01/12/nyregion/uber-lyft-juno-ride-hailing.html}}: Provides some information on average earnings of drivers per hours, problems like deciding schedules, etc.