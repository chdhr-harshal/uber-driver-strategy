%!TEX root = main.tex

\section{Related Work}
\label{sec:related_work}

To the best of our knowledge, we are the first to formally address the problem of optimizing
the driver's strategy in ridesharing platforms like Lyft and Uber. 
Interestingly, there have been recent articles in the popular press as well as blogposts on 
offering (often contradictory) advice to ridesharing drivers how to maximize earnings, mostly via chasing surge~\cite{dont,tips}.
Apart from that, the only relevant existing technical work 
studies other aspects of ridesharing platforms like Lyft and Uber. 
Also related can be considered existing work on optimization problems related to taxi routes. We discuss existing works along these lines next.


\spara{Studies of ridesharing platforms:}
Recent work has investigated the supply-side effects of specific incentives that Uber and Lyft provide to drivers, notably the impact of surge pricing~\cite{slaves}.  For example, Chen and Sheldon~\cite{chen2016dynamic} showed 
a causal relationship that drivers on Uber respond to surges by driving more during high surge times,  differentiating from previous work that suggests taxi drivers primarily focus on achieving earnings goals~\cite{camerer1997labor}. 
Other work has viewed ridesharing platforms more holistically, at a macro level.
Chen {\etal}~\cite{chen2015peeking} measured many facets of Uber in NYC, including the prevalence and extent 
  of surge pricing.
Hall and Krueger~\cite{hall2016analysis} showed that drivers were attracted to Uber platform due to flexibility it offers, 
and the level of compensation, but that earnings per hour do not vary much with number of hours worked. 
All these studies perform an {\em a posteriori} analysis of the data
and they do not focus on devising specific recommendations for drivers as we do.


 In another line of work, Banerjee {\etal}~\cite{banerjee2015pricing} studied  
 dynamic
pricing strategies for ridesharing platforms (such as Lyft) using a 
queuing-theoretic economic model. 
They showed that dynamic pricing is robust to changes in system parameters, even if it does not 
  achieve higher performance than static pricing.  
More recently, Ozkan and Ward\cite{ozkan2016dynamic} looked at strategic matching between supply (individual drivers) 
  and demand (requested rides) for Uber and Lyft.
They showed that matching based on parameters like driver and customer arrival rates, 
willingness of customers to wait and time-variance can achieve better performance than naively matching 
passengers with the closest driver. 
Although these works build interesting models for ridesharing economies, they are
orthogonal to ours as they take a more holistic view of such economies, while we choose to focus 
 on earnings of individual, self-interested drivers.




\spara{Optimization problems for taxi fleets:}
A considerable body of related work has focused on the optimization of taxi fleets, for
example building economic network models to describe demand and supply equilibria of taxi 
services under various tariff structures, fleet size regulations and other policy
alternatives~\cite{bailey1987simulation,yang2002demand}.  More recent work seeks to 
maximize social welfare by optimizing allocation of taxi market resources~\cite{shi2016optimization}.
Another direction relates to route optimization by a centralized administrator, e.g., in
the context of online and offline taxi dispatching services~\cite{maciejewski2013simulation,nunes2011taxi} 
and in the case of maximizing occupancy and minimizing travel times (by minimizing passenger detours) 
in a shared-ride setting~\cite{jung2013design}.
Other work has studied the supply side of the driving market from the viewpoint of behavioral economics.
A seminal paper by ~\cite{camerer1997labor} studied cabdrivers and found that  (1) `inexperienced' 
cabdrivers make labor supply decisions ``one day at a time'' instead of substituting labor and leisure 
across multiple days, and  (2) set a loose daily income target and quit working once they reach that 
target.  
Although related, all these works do not focus on the design of a specific gain-optimizing
strategy for drivers, as we do.

\begin{comment}

\textbf{Works on general taxi cabs:}

2. \cite{yang1998network} - first in the series of works modeling taxi utilization and movement to find that higher utilization leads to longer waiting times for customers.

3. \cite{wong2001modeling} - second paper, extends previous paper to incorporate effects of congestion and customer demand elasticity. 

4. \cite{yang2002demand} - third paper, uses network model to describe demand and supply equilibrium of taxi services under fare structure and fleet size regulation in competitive / monopoly market.

5. \cite{bailey1987simulation} - directed at understanding the dynamic interaction between demand, service rates and policy alternatives. Finds that customer waiting time is insensitive to changes in demand, but highly sensitive to changes in taxi fleet size.

6. \cite{qin2017mining} - explore the factors affecting driver incomes with quantitative estimates using GPS traces of over 167 million trips in Shanghai.

7. \cite{rong2016rich} - MDP to increase the revenue per unit time of taxi drivers. They study the relocate action from our strategy.

\textbf{Works on taxi routing:}

8. \cite{maciejewski2013simulation} - optimizes taxi routing by generating demand and congested network simulation. Defines online and offline taxi dispatching strategies and evaluate them. `No-scheduling' strategy works well under low demand but deteriorates under heavy load.

9. \cite{nunes2011taxi} - formulates a TSP problem to find best route for a taxi company to satisfy demand.

\textbf{Works on ride-sharing:}

10. \cite{agatz2012optimization} - outline optimization challenges in developing technology to support ride-sharing services. Survey of operations research papers in the domain.

11. \cite{santos2013dynamic} - Prove that problem of maximizing shared trips within a fixed time window to minimize shared expenses is NP-Hard and a propose heuristic solution.

12. \cite{jung2013design} - Simulated Annealing algorithm to maximize occupancy and minimize travel times (by minimizing passenger detours) in shared-ride concept.

\textbf{Works on ride-sharing vehicle routing:}

13. \cite{lin2012research} - simulated annealing algorithm to optimize routing of ride-sharing taxi to minimize operating costs while maximizing customer satisfaction.

\textbf{Strategic behavior:}

14. \cite{shi2016optimization} - maximizes social welfare and optimizes allocation of taxi market resources. Also analyzes strategic behavior of passengers who may join or drop out of system based on their social welfare threshold.

\textbf{Uber related works:}

15. \cite{hall2016analysis} - Drivers attracted to Uber platform due to flexibility it offers, level of compensation, earnings per hour do not vary much with number of hours worked. 

16. \cite{chen2016dynamic} - Show a causal relationship that drivers on Uber respond to `surges' by driving more during high surge times, in contrast to \cite{camerer1997labor} which says that drivers driver until they achieve earnings goals.

17. \cite{banerjee2015pricing} - Study complex dynamic pricing strategies for ride-sharing platforms (Lyft) using a queuing-theoretic economic model. They show the dynamic pricing is robust to changes in system parameters, even if it does not achieve higher performance than static pricing. 

18. \cite{ozkan2016dynamic} - Strategic matching between Uber or Lyft's supply and demand. Matching based on parameters like customer/driver arrival rates, willingness of customers to wait and time-variance can achieve better performance than naively matching passenger with closest driver.

\textbf{Media and Press articles:}


 

%%3. 
\end{comment}
